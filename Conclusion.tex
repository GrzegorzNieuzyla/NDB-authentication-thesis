% !TeX spellcheck = pl_PL
\chapter{Podsumowanie}
\label{chp:conclusion}
Analiza dostępnych algorytmów generacji Negatywnych Baz Danych miała na celu znalezienie rozwiązania umożliwiającego zaimplementowanie bezpiecznego systemu uwierzytelniania użytkownika.
Głównym założeniem było istnienie negatywnej reprezentacji informacji, która zagwarantuje że pierwotna postać danych będzie trudna do odzyskania.

Po przetestowaniu odnalezionych w~literaturze metod generacji $NDB$ udało się znaleźć odpowiedni algorytm oraz kombinację parametrów tworzących bazy, które nie są łatwo odwracalne przez solwery SAT. 
Pozwoliło to na zaimplementowanie aplikacji przechowującej skróty haseł w~formie negatywnej i~oferującej wszystkie funkcje oczekiwane od systemu uwierzytelniania. 

W~ramach niniejszej pracy zrealizowane zostały testy $NDB$ wygenerowanych przez poszczególne metody generacji (algorytm prefiksowy, \textit{Randomize\_NDB}, \textit{Q-Hidden}, \textit{K-Hidden} i~algorytm hybrydowy) pod kątem trudności odwrócenia. 
Na ich podstawie stworzona została w~pełni funkcjonalna propozycja systemu uwierzytelniania oferująca interfejs graficzny i podstawowe funkcjonalności zarządzania kontem użytkownika.

\section{Bezpieczeństwo przedstawionej implementacji}
Z~testów przedstawionych w~rozdziale \ref{chp:tests} wynika, że istnieją algorytmy $NDB$ pozwalające na generowanie trudnych formuł SAT, które są rozwiązywalne w czasie wykładniczym.
Zostało to wykorzystane podczas implementacji aplikacji umożliwiającej uwierzytelnianie użytkowników za pomocą loginu i~hasła. Użycie $NDB$ zamiast pozytywnej bazy danych do składowania
skrótu hasła nie zwiększa drastycznie bezpieczeństwa w~stosunku do dobrze zaprojektowanego systemu, używającego odpowiednich procedur kryptograficznych, w~szczególności algorytmu uzyskiwania klucza
na podstawie hasła. Jednak obecność dodatkowej warstwy może odeprzeć atak na tak zaprojektowaną aplikację, gdy pozostałe metody zawiodą.

Sam fakt użycia $NDB$ nie rozwiąże głównych problemów z systemami uwierzytelniania, zwłaszcza stosowania słabych i~przewidywalnych haseł, dlatego w~implementacji aplikacji wymuszam wybór fraz o~zwiększonej złożoności.
Dodatkowo konieczność porównania klucza z~rekordami negatywnym może spowolnić próby złamania przez metody \textit{brute-force}.

Negatywne bazy danych mogą oferować znacznie większą poprawę bezpieczeństwa w~systemach, w~których nie jest konieczne powiązanie rekordów z~dodatkowymi danymi.
W~tym celu należy zrezygnować z~większości zalet posiadanych przez struktury przechowywania danych w~postaci pozytywnej i~zmodyfikować problem do takiego, który wykorzystuje tylko jedną operację -- sprawdzenie obecności danego rekordu w~bazie.

\section{Możliwe rozszerzenia}
Przedstawiona aplikacja, poza dodaniem referencji do dodatkowych danych opisujących użytkownika może być dowolnie zmodyfikowana w~celu spełnienia określonych wymagań biznesowych. 

Jedną z opcji dalszego zwiększenia bezpieczeństwa może być zaimplementowanie 
uwierzytelniania dwuskładnikowego, uniemożliwiającego dostęp do konta w przypadku uzyskania dostępu do tylko jednego składnika. Żeby zapewnić stosowanie dobrych nawyków podczas korzystania z~aplikacji możliwe jest także stworzenie modułu wymuszającego częste zmiany hasła.