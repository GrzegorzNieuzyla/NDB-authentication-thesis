% !TeX spellcheck = pl_PL
\chapter{Wprowadzenie}
\section{Cele pracy}
Celem niniejszej pracy jest implementacja i przetestowanie systemu uwierzytelniania oferującego większe bezpieczeństwo
niż standardowy schemat generowania skrótu hasła za pomocą funkcji generacji klucza (np. PBKDF2, bcrypt) i przechowywaniu 
w standardowej (pozytywnej) bazie danych.

Założeniem systemu jest zamienienie reprezentacji w sposób jawny na Negatywną Bazę Danych (NDB) co pozwoli dodać dodatkową warstwę bezpieczeństwa
która znacząco utrudni uzyskanie haseł użytkownika w przypadku wykradzenia bazy danych.

Rezultatem wyjściowym algorytmów generacji NDB jest zbiór formuł logicznych, dlatego przeprowadzone zostały testy z wykorzystaniem solwerów SAT mające na celu zasymulowanie ataku na powstałą NDB.

\section{Zawartość pracy}

















