% !TeX spellcheck = pl_PL
\chapter{Wprowadzenie}
\label{chp:intro}
\section{Cele pracy}
Jak wynika z~raportu Google\cite{google-poll}, większość internautów ma problem z~zachowaniem zasad bezpieczeństwa dotyczącego 
używania mediów społecznościowych i~portali internetowych. $66\%$ procent badanych przyznało się do używania jednego hasła do różnych kont. 
W~przypadku wykradzenia bazy danych pojedynczego portalu i~uzyskaniu postaci jawnych haseł istnieje ryzyko uzyskania dostępu również do innych aplikacji
przez atakującego. W~takiej sytuacji wyciek danych z~amatorskiego forum tematycznego może zagrozić naszemu kontu bankowemu lub poczcie elektronicznej.

Główną przyczyną nadużywania jednego hasła jest konieczność zapamiętania danych logowaniu do wielu stron internetowych -- $75\%$ internautów ma problem z~ich zarządzeniem.
  
Istnieje szereg metod, które indywidualny użytkownik może zastosować w~celu polepszenia swojego bezpieczeństwa w~Internecie, ale nie cieszą
się one dużą popularnością. Według raportu tylko $37\%$ internautów używa uwierzytelnienia wielopoziomowego, a~menadżerów haseł
umożliwiających stosowanie różnych i~silnych danych logowania -- $15\%$.

Częściowym rozwiązaniem tego problemu może być wprowadzenie dodatkowych warstw bezpieczeństwa bezpośrednio w~systemach uwierzytelniania
w~celu utrudnienia uzyskania nieautoryzowanego dostępu do wrażliwych danych. Takie podejście pozwoli podnieść bezpieczeństwo wszystkich użytkowników,
niezależnie od indywidualnych preferencji i~nawyków.

Jedną z~możliwości zwiększenia bezpieczeństwa jest użycie Negatywnej Bazy Danych ($NDB$), reprezentującej ukrytą informację w~postaci jawnej negatywnej.
Odzyskanie tak zakodowanych danych jest problemem o~złożoności SAT.
\\
Celem niniejszej pracy jest:
\begin{itemize}
    \item Wykonanie testów algorytmów generacji Negatywnych Baz Danych z~zastosowaniem solwerów SAT.
    \item Implementacja systemu uwierzytelniania oferującego większe bezpieczeństwo
    niż standardowy schemat generowania skrótu hasła za pomocą funkcji generacji klucza (np. \textit{PBKDF2}, \textit{bcrypt})
    i~przechowywaniu w~standardowej (pozytywnej) bazie danych.
\end{itemize} 
 
 
Założeniem systemu jest zamienienie reprezentacji w~sposób jawny na Negatywną Bazę Danych, co pozwoli dodać 
dodatkową warstwę bezpieczeństwa która znacząco utrudni uzyskanie haseł użytkownika w~przypadku kradzieży bazy danych.

W~tym celu opisałem różne algorytmy prezentowane w~dostępnej literaturze, przedstawiłem schemat ich działania i~porównałem
je pod względem bezpieczeństwa. 

Rezultatem wyjściowym algorytmów generacji $NDB$ jest zbiór ciągów tekstowych, które można jednoznacznie sprowadzić 
do zbioru formuł logicznych CNF, dlatego przeprowadzone zostały testy z~wykorzystaniem solwerów SAT mające na celu 
zasymulowanie ataku na powstałą $NDB$.

Koncept negatywnej reprezentacji danych wywodzi się z~immunologii. Badania wykazały, że limfocyty $T$~wykorzystują
podobny proces w~celu rozróżnienia między sobą i~innymi komórkami \cite{T-cells}.

\section{Zawartość pracy}
Niniejsza praca jest podzielna na sześć rozdziałów. W~rozdziale \ref{chp:intro} przedstawiam obecne problemy występujące w~systemach uwierzytelniania i~wstępnie opisuję schemat działania Negatywnych Baz Danych.

W~rozdziale \ref{chp:sat-theory} opisuję problem spełnialności oraz metody i~techniki stosowane przez współczesne solwery SAT.

Rozdział \ref{chp:theory} poświęcony jest dostępnym w~literaturze algorytmom generacji Negatywnych Baz Danych. Porównywane są pod kątem teoretycznym na podstawie przytoczonych artykułów.

Dokładne testy metod generacji $NDB$ są przedstawione w~rozdziale \ref{chp:tests}. Opisany tam jest wpływ wszystkich dostępnym parametrów na trudność ich odwrócenia za pomocą solwerów SAT zChaff oraz WalkSAT.
Dodatkowo porównywane są metody zapobiegania występowaniu niepożądanych rekordów pozytywnych w~generowanej $NDB$.

W rozdziale \ref{chp:implementation} opisuję stworzoną na potrzeby tej pracy implementację systemu uwierzytelniania wykorzystującą Negatywne Bazy Danych.

Wnioski wyniesione z~analizowania konceptu $NDB$ i~pisania tej pracy znajdują się w~rozdziale \ref{chp:conclusion}.















