% !TeX spellcheck = pl_PL
\chapter{Wprowadzenie}
\section{Cele pracy}
Celem niniejszej pracy jest implementacja i~przetestowanie systemu uwierzytelniania oferującego większe bezpieczeństwo
niż standardowy schemat generowania skrótu hasła za pomocą funkcji generacji klucza (np. \textit{PBKDF2}, \textit{bcrypt})
i~przechowywaniu w~standardowej (pozytywnej) bazie danych.

Założeniem systemu jest zamienienie reprezentacji w~sposób jawny na Negatywną Bazę Danych ($NDB$) co pozwoli dodać 
dodatkową warstwę bezpieczeństwa która znacząco utrudni uzyskanie haseł użytkownika w~przypadku wykradzenia bazy danych.

W~tym celu opisałem różne algorytmy prezentowane w~dostępnej literaturze, przedstawiłem schemat ich działania i~porównałem
je pod względem bezpieczeństwa. 

Rezultatem wyjściowym algorytmów generacji NDB jest zbiór ciągów tekstowych, które można jednoznacznie sprowadzić 
do zbioru formuł logicznych CNF, dlatego przeprowadzone zostały testy z~wykorzystaniem solwerów SAT mające na celu 
zasymulowanie ataku na powstałą NDB.

\section{Zawartość pracy}

















