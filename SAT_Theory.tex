% !TeX spellcheck = pl_PL
\chapter{Solwery SAT}
\label{chp:sat-theory}
\section{Problem spełnialności}
Problem spełnialności (ang. \textit{Boolean satisfiability problem}, w~skrócie SAT) polega na sprawdzeniu, czy dana formuła logiczna
posiada odpowiednie przypisanie zmiennych przy których całość wyrażenia będzie ewaluowana do prawdy.

Wyrażenia wykorzystywane w~problemie SAT składają się z: 
\begin{itemize}
    \item zmiennych - $x_1, x_2 \dots x_n$, mogących przybrać wartość prawdziwą ($true$, $T$, $\top$, $1$) 
    lub fałszywą ($false$,~$F$,~$\bot$,~$0$)
    \item operatora koniunkcji - $AND$, $\land$
    \item operatora alternatywy - $OR$, $\lor$
    \item operatora negacji - $NOT$, $\lnot$
    \item nawiasów - $()$
\end{itemize}

Formuły SAT przedstawia się w~postaci CNF (koniunkcyjna postać normalna, ang. \textit{conjunctive normal form}) 
której struktura jest następująca: formuła jest koniunkcją klauzul, a~klauzula jest alternatywą literałów (tj. zmiennych lub ich negacji):
\[ (x_{11} \lor x_{12} \lor \dots \lor x_{1n} ) \land (x_{21} \lor x_{22} \lor \dots \lor x_{2n} ) \land  \dots \land (x_{m1} \lor x_{m2} \lor \dots \lor x_{mn} ) \]

Naiwny algorytm SAT jest trywialny -~wystarczy dla każdej możliwej kombinacji zmiennych $x_1 \dots x_n$
sprawdzić czy formuła jest spełnialna. Sprawdzenie następuje w~czasie liniowym od wartości $n$, zatem cały algorytm
ma złożoność obliczeniową $O(2^n)$ -~nie do przyjęcia w~praktycznie żadnych zastosowaniach poza bardzo małymi wartościami $n$.

Problem SAT jest pierwszym problemem dla którego udowodniono że jest $NP$-zupełny (Twierdzenie Cooka-Levina \cite{cook-SAT, levin-SAT}).
Oznacza to, że można w~czasie wielomianowych sprowadzić każdy problem decyzyjny zawierający się w~$NP$ do SAT.

Problem należący do $NP$ charakteryzuje się tym, że zweryfikowanie pojedynczego rozwiązania jest możliwe w~czasie wielomianowym
dla deterministycznej Maszyny Turinga, oraz że sprawdzenie wszystkich możliwości (a~więc rozstrzygnięcie problemu) jest możliwe
również w~czasie wielomianowym, ale dla niedeterministycznej Maszyny Turinga -~przez jednoczesne sprawdzenie każdej kombinacji. 
Wynika z~tego również, że $P \subseteq NP$.

Nie każda możliwa formuła CNF stanowi problem $NP$-zupełny. Twierdzenie o~dychotomii Schaefera określa 
szczególne instancje problemu należące do $P$ (przy założeniu, że $P \neq NP$) \cite{schaefer-dichotomy}:

\begin{enumerate}
    \item Formuła jest spełnialna jeśli wszystkie zmienne przyjmują wartość 0
    \item Formuła jest spełnialna jeśli wszystkie zmienne przyjmują wartość 1
    \item Każda klauzula ma co najwyżej jeden literał pozytywny (każda klauzula jest klauzulą Horna)
    \item Każda klauzula ma co najwyżej jeden literał negatywny (każda klauzula jest dualną klauzulą Horna)
    \item Każda klauzula ma co najwyżej dwa literały (2-CNF)
    \item Formuła jest równoznaczna systemowi równań linowych $x_1 \oplus x_2 \oplus \dots \oplus x_n = c$, gdzie
    $x_i,~c~\in~\{0, 1\}$ a~$\oplus \equiv$ dodawanie $mod_2$
    
\end{enumerate}

\section{Opis działania solwerów SAT}
Solwery SAT można podzielić na dwie kategorie -- kompletne i~niekompletne.
Algorytmy kompletne gwarantują, że jeśli istnieje spełniające przypisanie dla analizowanej formuły to zostanie zwrócony wynik pozytywny.
Natomiast wynik działania solwera niekompletnego może być albo pozytywny -- istnieje odpowiednie przypisanie, albo niezdefiniowany -- algorytm nie
znalazł rozwiązania w~określonej liczbie prób.

\subsection{Solwery kompletne}
Algorytmy kompletne do rozstrzygania problemów SAT korzystają z~szeregu metod do których należą: 
kwantyfikacja egzystencjalna, wnioskowanie, przeszukiwanie oraz wnioskowanie i~przeszukiwanie \cite{handbook-satifiability-complete}.

\subsubsection{Kwantyfikacja egzystencjalna}
Kwantyfikacja egzystencjalna jest klasą algorytmów upraszczających zadaną formułę CNF do postaci spełnialnej $\{\}$ lub niespełnialnej $\{\emptyset\}$.
W tym celu definiuje się operator $\exists$ zmiennej $P$ w~formule $\Delta$:
\[ \exists P \Delta \stackrel{\text{def}}{=} (\Delta | P) \lor (\Delta | \neg P) \]
gdzie operator '$|$' jest \textit{warunkowaniem} i~definiuje się je następująco:
\[ \Delta | P = \{\alpha - \{\neg P\}~|~\alpha \in \Delta, P~\notin \alpha\} ~~\text{\cite{handbook-satifiability-complete}} \] 

Proces wnioskowania można intuicyjnie przedstawić jako ustawienie zmiennej P~na wartość prawdziwą (lub fałszywą w~przypadku $\neg P$)
i~zredukowanie zbioru klauzul usuwając te, które są spełnione oraz usuwając z~pozostałych literały mające wartość fałszywą.
Kwantyfikator $\exists P~\Delta $ określa formułę po wyeliminowaniu zmiennej $P$ przez rozpatrzenie obu możliwych wartości.

Z~punktu widzenia problemu spełnialności najważniejszą właściwością jest to, że formuła $\Delta$ jest spełnialna wtedy i~tylko wtedy, gdy 
$\exists P~\Delta$ jest spełnialna. Pozwala to na stopniowe eliminowanie zmiennych aż do osiągnięcia pustej klauzuli lub pustej formuły. 
Kwantyfikacja egzystencjalna jest wykorzystywana m.in w~algorytmie DP oraz w~symbolicznym rozwiązywaniu SAT.

\subsubsection{Wnioskowanie}
Upraszczanie formuł przez wnioskowanie opiera się na usuwaniu redundantnych literałów tj. wykluczających się nawzajem.
Przykładowo w~formule:
\[ \{\{x_1, x_2, \neg x_3 \},  \{ \neg x_2, x_4, x_5 \}\} \]
zmienna $x_2$ występuje w~postaci zanegowanej i~niezanegowanej w~różnych klauzulach. Oznacza to, żeby cała formuła była spełniona,
co najmniej jeden pozostały literał musi się ewaluować do prawdy. Można zatem zastosować następujące przekształcenie:
  \[ \{\{x_1, \neg x_3, x_4, x_5 \}\} \]
  
Dana formuła jest niespełnialna wtedy i~tylko wtedy, gdy istnieje wnioskowanie prowadzące do pustej klauzuli, zatem po wyczerpaniu
wszystkich możliwości wnioskowania możemy ustalić, że problem jest spełnialny. 

Do algorytmów wykorzystujących tę metodę należą algorytm St{\aa}lmarcka i~HeerHugo.

\subsubsection{Przeszukiwanie}
Problem SAT można przedstawić jako drzewo decyzyjne przypisania zmiennych, gdzie liście są wszystkimi możliwymi kombinacjami literałów.
Metoda przeszukiwania polega na znalezieniu liścia spełniającego formułę. Używa się do tego algorytmu przeszukiwania wszerz oraz warunkowania
w~celu ograniczenia drzewa. 

Dodatkowo używa się także \textit{propagacji jednostkowej} (ang. \textit{unit propagation}) za pomocą której
można odrzucić lub potwierdzić poprawność danego poddrzewa we wczesnej fazie analizy przez spełnienie klauzul jednostkowych (posiadających tylko jeden literał).

Wzorcowym zastosowaniem tej metody jest algorytm \textit{DPLL}.

\subsubsection{Wnioskowanie i przeszukiwanie}
Wadą podejścia zastosowanego w~\textit{DPLL} jest fakt, że podczas cofania się po drzewie w~kierunku korzenia nie bierze się pod uwagę
zbioru zmiennych, które spowodowały odrzucenie danego poddrzewa.
Dlatego w~tej metodzie stosuje się tzw. niechronologiczne cofanie -- podczas konfliktu literałów wyznacza się zbiór powodujących 
go zmiennych i~zmienia się ich wartości, pomijając te nie biorące udziału w~konflikcie.

Dodatkowo podczas działania algorytmu wprowadza się dodatkowe klauzule przedstawiające zależności, które pozwalają wykrywać nieścisłości 
bliżej korzenia, jeszcze bardziej skracając czas działania.

Do tej klasy należy większość współczesnych, najbardziej optymalnych algorytmów kompletnych rozstrzygania problemu SAT -- m.in 
\textit{zChaff}, \textit{MiniSAT}, \textit{PicoSAT} i~\textit{Siege}.

\subsection{Solwery niekompletne}
Solwery nie gwarantujące znalezienia rozwiązania opierają się na \textit{stochastycznym przeszukiwaniu lokalnym}.
Takie rozwiązanie pozwala na znaczne mniejsze czasy wykonywania dla odpowiednich formuł niż algorytmy wykorzystujące DPLL, ale
w~przypadku trudnych problemów zamiast zdegenerowania złożoności obliczeniowej do $O(2^n)$ zostaje zwrócony wynik nieokreślony.  

Czołowymi solwerami z~tej kategorii są \textit{GSAT} i~\textit{WalkSAT} \cite{handbook-satifiability-incomplete}.


\subsubsection{\textit{GSAT}}
\label{sec:gsat}
Schemat działania \textit{GSAT} polega na wygenerowaniu losowego przypisania i~zachłannym \textit{odwracaniu} (ang. \textit{flipping})
wartości zmiennych w~celu zminimalizowania liczby niespełnionych klauzul. Ilość odwróceń oraz liczbę prób
z~wygenerowaniem nowego ciągu początkowego można modyfikować za pomocą odpowiednich parametrów.

\subsubsection{\textit{WalkSAT}}
\textit{WalkSAT} rozszerza algorytm stopniowego odwracania zmiennych używany w~\textit{GSAT}, redukując stopień zachłanności.
Na początku każdej iteracji wybierana jest losowa klauzula, i~w~niej szukana jest zmienna, której zanegowanie nie zmieni żadnej
innej klauzuli na niespełnioną. Jeśli nie została znaleziona, \textit{WalkSAT} wybiera z~danej klauzuli losową zmienną lub taką, której
zmiana spowoduję spełnienie jak największej liczby pozostałych klauzul. Wprowadzony jest dodatkowo parametr szumu $p \in [0, 1]$ określający
jak często wybierane jest podejście nie-zachłanne. Dla losowych formuł 3SAT optymalna wartość szumu wynosi $p~=~0.57$ \cite{handbook-satifiability-incomplete}.

