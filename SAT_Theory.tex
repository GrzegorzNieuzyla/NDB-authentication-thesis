% !TeX spellcheck = pl_PL
\chapter{Solwery SAT}

\section{Problem spełnialności}
Problem spełnialności (ang. \textit{Boolean satisfiability problem}, w~skrócie SAT) polega na sprawdzeniu, czy dana formuła logiczna
posiada odpowiednie przypisanie zmiennych przy których cała formuła będzie prawdą.

Wyrażenia wykorzystywane w~problemie SAT składają się z: 
\begin{itemize}
    \item zmiennych - $x_1, x_2 \dots x_n$, mogących przybrać wartość prawdziwą ($true$, $T$, $\top$, $1$) 
    lub fałszywą ($false$,~$F$,~$\bot$,~$0$)
    \item operatora koniunkcji - $AND$, $\land$
    \item operatora alternatywy - $OR$, $\lor$
    \item operatora negacji - $NOT$, $\lnot$
    \item nawiasów - $()$
\end{itemize}

Formuły SAT przedstawia się w~postaci CNF (koniunkcyjna postać normalna, ang. \textit{conjunctive normal form}) 
której struktura jest następująca: formuła jest koniunkcją klauzul, a~klauzula jest alternatywą literałów (tj. zmiennych lub ich negacji).

\[ (x_{11} \lor x_{12} \lor \dots \lor x_{1n} ) \land (x_{21} \lor x_{22} \lor \dots \lor x_{2n} ) \land  \dots \land (x_{m1} \lor x_{m2} \lor \dots \lor x_{mn} ) \]


Naiwny algorytm SAT jest trywialny - wystarczy dla każdej możliwej kombinacji zmiennych $x_1 \dots x_2$
sprawdzić czy formuła jest spełnialna. Sprawdzenie następuje w~czasie liniowym od wartości $n$, zatem cały algorytm
ma złożoność obliczeniową $O(2^n)$ - niepraktyczną w praktycznie każdych zastosowaniach poza bardzo małymi wartościami $n$.

Problem SAT jest pierwszym problemem dla którego udowodniono że jest $NP$-zupełny (Twierdzenie Cooka-Levina \cite{cook-SAT, levin-SAT}).
Oznacza to, że można w czasie wielomianowych sprowadzić każdy problem decyzyjny zawierający się w $NP$ do SAT.

Problem należący do $NP$ charakteryzuje się tym, że zweryfikowanie pojedynczego rozwiązania jest możliwe w~czasie wielomianowym
dla deterministycznej Maszyny Turinga, oraz że sprawdzenie wszystkich możliwości (a~więc rozstrzygnięcie problemu) jest możliwe
również w~czasie wielomianowym, ale dla niedeterministycznej Maszyny Turinga - przez jednoczesne sprawdzenie każdej kombinacji. 
Wynika z~tego również, że $P \subseteq NP$.

Nie każda możliwa formuła CNF stanowi problem $NP$-zupełny. Twierdzenie o dychotomii Schaefera określa 
szczególne instancje problemu należące do $P$ (przy założeniu, że $P \neq NP$) \cite{schaefer-dichotomy}:

\begin{enumerate}
    \item Formuła jest spełnialna jeśli wszystkie zmienne przyjmują wartość 0
    \item Formuła jest spełnialna jeśli wszystkie zmienne przyjmują wartość 1
    \item Każda klauzula ma co najwyżej jeden literał pozytywny (każda klauzula jest klauzulą Horna)
    \item Każda klauzula ma co najwyżej jeden literał negatywny (każda klauzula jest dualną klauzulą Horna)
    \item Każda klauzula ma co najwyżej dwa literały (2-CNF)
    \item Formuła jest równoznaczna systemowi równań linowych $x_1 \oplus x_2 \oplus \dots \oplus x_n = c$, gdzie
    $x_i,~c~\in~\{0, 1\}$ a~$\oplus \equiv$ dodawanie $mod_2$
    
\end{enumerate}

\section{Opis działania solwerów SAT}
Solwery SAT można podzielić na dwie kategorie -- kompletne i niekompletne.
Algorytmy kompletne gwarantują, że jeśli istnieje spełniające przypisanie dla analizowanej formuły to zostanie zwrócony wynik pozytywny.
Natomiast wynikiem działania solwera niekompletnego jest albo pozytywny -- istnieje odpowiednie przypisanie lub niezdefiniowany -- algorytm nie
znalazł rozwiązania w określonej liczbie prób.

\subsection{Solwery kompletne}
Algorytmy kompletne do rozstrzygania problemów SAT korzystają z szeregu metod do których należą: 
kwantyfikacja egzystencjalna, wnioskowanie, przeszukiwanie, wnioskowanie i przeszukiwanie \cite{handbook-satifiability-complete}.
\subsubsection{Kwantyfikacja egzystencjalna}
\subsubsection{Wnioskowanie}
\subsubsection{Przeszukiwanie}
\subsubsection{Wnioskowanie i przeszukiwanie}
\subsection{Solwery niekompletne}
Do metod niekompletnego rozstrzygania spełnialności należą:  \cite{handbook-satifiability-incomplete}. 

\section{Wykorzystywanie solwerów SAT w celu uzyskania przeciwobrazu Negatywnej Bazy Danych}

Problem znalezienia ukrytego ciągu w Negatywnej Bazie Danych jest równoważny SAT i zapis napisów nad alfabetem 
$\{0, 1, *\}$ można sprowadzić do CNF w czasie liniowym.
 Dla przykładu, poniższy zbiór rekordów można zinterpretować jako 
 \enquote{Znajdź takie przypisanie wartościami \{0, 1\}, żeby dla każdego rekordu nie pokrywało się na co najmniej jednej pozycji}.
 
\begin{table}[h]

    \centering
    \label{Tbl:NDB-sat-example}
    \begin{tabular}{|l|}
    	\hline
    	11*0 \\ \hline
    	001* \\ \hline
    	*111 \\ \hline
    	*101 \\ \hline
    \end{tabular}
    \caption{Przykładowy zbiór rekordów $NDB$}
\end{table}

Zatem pomijając symbole $*$, zamieniając wartości $\{0,1\}$ na odpowiadające literały, negując je i łącząc operatorem alternatywy
oraz wstawiając operatory koniunkcji pomiędzy powstałe klauzule otrzymujemy postać CNF:

\[ (\neg x_1 \lor \neg x_2 \lor x_4)~\land \]\[  (x_1 \lor x_2 \lor \neg x_3)~\land \]\[ (\neg x_2 \lor \neg x_3 \lor \neg x_4)~\land \]\[(\neg x_2 \lor x_3 \lor \neg x_4) \]


Metoda działania solwerów SAT nie polega znalezieniu wszystkich rozwiązań tylko na sprawdzeniu czy dana formuła jest spełnialna, 
dlatego w sytuacji próby odwrócenia baz wielorekordowych lub wygenerowanych z użyciem algorytmów niekompletnych atakujący potrzebuje dla każdego pojedynczego rozwiązania dodać dodatkową formułę przeczącą znalezionemu rekordowi i ponownie uruchomić solwer.   

W związku z faktem, że uzyskanie dostępu do nawet pojedynczego rekordu w systemach uwierzytelniania jest niedopuszczalne, testy są przeprowadzone pod kątem znalezienia pierwszego rozwiązania. 