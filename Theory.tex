% !TeX spellcheck = pl_PL
\chapter{Negatywne Bazy Danych - opis teoretyczny}
\section{Opis działania}
Główną operacja wykonywalną na NDB jest sprawdzenie czy dany rekord znajduje się w bazie. Przyjmując $U$ 
jako oznaczenie uniwersum języka binarnego o długości $l$ a $DB$ jako zbiór wszystkich rekordów, każdy o długości $l$,
$NDB$ przechowuje zbiór $U - DB$ \cite{NRI-Esponda}. Takie dane są niepraktycznie do zareprezentowania w postaci nieskompresowanej z uwagi na wielkość, dlatego
stosuje się wyrazy nad alfabetem $\{0,1,*\}$ gdzie symbol $*$ może oznaczać zarówno $0$ lub $1$ w jawnej reprezentacji bitowej.
Każdy taki wyraz odpowiada jednemu lub wielu elementom $U - DB$ i jest sprowadzany do formuły logicznej (Tabela \ref{Tbl:NDB-logform}).
Z założenia algorytm sprawdzający przynależność do $DB$ sprawdza czy jakakolwiek formuła z NDB jest spełniana przez dany rekord. 
Dane znajdują się w $DB$ wtedy i tylko wtedy gdy żadna formuła nie zostanie spełniona. Taki model działania wymusza na danych stałą wielkość,
co jednak nie stanowi problemu w przypadku przechowywania skrótów haseł które mają stałą, zależną od konkretnego algorytmu długość.

\begin{table}[h]
    \caption{Reprezentacja formuł logicznych za pomocą NDB}
    \centering
    \label{Tbl:NDB-logform}
    \begin{tabular}{|l|l|}
        \hline
        rekord NDB & formuła logiczna                       \\ \hline
        011*       & $\neg{x_1} \land x_2 \land x_3$        \\ \hline
        0*01       & $\neg{x_1} \land \neg{x_3} \land x_4 $ \\ \hline
        111*       & $x_1 \land x_2 \land x3$               \\ \hline
    \end{tabular}
\end{table}

\section{Zastosowanie w systemach uwierzytelniania}
NDB może być wykorzystana w każdym systemie, gdzie podstawową operacją na danych jest sprawdzenie czy
dany rekord znajduje się w bazie. Jednym z najpopularniejszych systemów uwierzytelniania jest metoda oparta na loginie i haśle.
Użytkownik danej aplikacji przy zakładaniu konta podaje hasło, które następnie warstwa serwerowa danej aplikacji przechowuje jako wynik nieodwracalnej funkcji hashującej.

W przypadku nieautoryzowanego dostępu do bazy danych i używanego algorytmu uzyskiwania skrótu z hasła, atakujący może uzyskać 
wartość pierwotną mało skomplikowanych haseł za pomocą np. metody słownikowej. Modyfikując powyższy algorytm 
składując skróty jako rekordy w NDB uniemożliwiamy iterację wszystkich danych, jednocześnie pozostawiając
łatwy dostęp do informacji czy użytkownik o podanym loginie i haśle ma dostęp do aplikacji.

\section{Algorytmy generowania Negatywnych Baz Danych}
\subsection{Algorytm prefiksowy}
Najprostszym ze sposobów generowania Negatywnych Baz Danych jest zaproponowany przez Fernando Esponda algorytm prefiksowy \cite{NRI-Esponda, Esponda2004EnhancingPT}. Został on opracowany w celu udowodnienia że proces generowania NDB z rekordów $DB$ jest możliwy w rozsądnej złożoności czasowej i pamięciowej.
\\\\\\
\begin{algorithm}[H]
    \SetAlgoLined
    
    \KwData{$DB$ - zbiór rekordów do zareprezentowania w NDB, $l$ - liczba rekordów w $DB$}
    \KwResult{Zbiór rekordów NDB}
    $Prefix_n(V)$ - Prefiks n-znakowy rekordu $V$\\
    $len(V)$ - Długość rekordu $V$\\
    $W_i$ = \{\};\\
    i = 0;\\
    \While{i < l}{
        $W_i$ = \{ $V_p$ $|$ $len(V_p)$ = i + 1, $V_p \notin$ \{$Prefix_{i+1}(V) | V \in DB$ \}  , $Prefix_i(V_p) \in  W_i$\}\\
        \For{$V_p$ in $W_i$}{
            Stwórz rekord NDB o długości $l$ którego $V_p$ jest prefiksem a na pozostałych pozycjach jest symbol $*$ i dodaj do zbioru wyjściowego
            
        }
        i = i + 1;\\
        $W_i$ = \{ $Prefix_i(V) | V \in DB$\}
    }
    
    \caption{Algorytm prefiksowy}
    \label{alg:prefix}
\end{algorithm}
~\\\\

\begin{explanation}
    Powyższa metoda polega na generowaniu coraz dłuższych prefiksów które nie pokrywają się ze zbiorem $DB$.
    W ten sposób na początku tworzone są rekordy odpowiadające znacznej części $U~-~DB$. Czasami występuje potrzeba zdefiniowania pewnych rekordów explicite bez wykorzystania symbolu $*$ jeżeli każdy możliwy prefiks jest także prefiksem rekordu z $DB$. 
    Przykładowy wynik działania znajduje się w tabeli \ref{tbl:prefix_results}.
    
    \begin{table}[h]
        \centering
        \begin{tabular}{|l|l|l|}
            \hline
            DB   & U - DB & NDB  \\ \hline
            0000 & 0001   & 10** \\
            0110 & 0011   & 010* \\
            0010 & 0100   & 111* \\
            1101 & 0101   & 0001 \\
            & 0111   & 0011 \\
            & 1000   & 0111 \\
            & 1001   & 1100 \\
            & 1010   &      \\
            & 1011   &      \\
            & 1100   &      \\
            & 1110   &      \\
            & 1111   &      \\ \hline
        \end{tabular}
        \caption{Rezultat działania algorytmu prefiksowego}
        \label{tbl:prefix_results}
    \end{table}
    
    
\end{explanation}
Algorytm prefiksowy jest deterministyczny i każdy powstały rekord reprezentuje unikalną, nie pokrywającą się część $U~-~DB$ \cite{NRI-Esponda}.
Powoduje to, że algorytm uzyskiwania zbioru $DB$ z otrzymanej NDB nie wymaga sprowadzenia do problemu SAT. Wystarczy jedynie odpowiednio posortować rekordy i wyznaczyć przedziały pomiędzy nimi.

\subsection{Algorytmy generujące trudne do odwrócenia Negatywne Bazy danych}
