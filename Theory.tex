% !TeX spellcheck = pl_PL
\chapter{Negatywne Bazy Danych - opis teoretyczny}
\label{chp:theory}
\section{Opis działania}
Główną operacja wykonywalną na $NDB$ jest sprawdzenie czy dany rekord znajduje się w bazie. Przyjmując $U$ 
jako oznaczenie uniwersum języka binarnego o~długości $l$ a $DB$ jako zbiór wszystkich rekordów, każdy o~długości $l$,
$NDB$ przechowuje zbiór $U - DB$ \cite{NRI-Esponda}. Takie dane są niepraktycznie do zareprezentowania w~postaci nieskompresowanej z~uwagi na wielkość, dlatego
stosuje się wyrazy nad alfabetem $\{0,1,*\}$ gdzie symbol $*$ może oznaczać zarówno $0$ lub $1$ w~jawnej reprezentacji bitowej.
Pozycje na których znajduje się wartość $0$ lub $1$ są \textit{ustalone} a z~wartością $*$ - \textit{nieustalone}.

Każdy taki wyraz odpowiada jednemu lub wielu elementom $U - DB$ i~jest sprowadzany do formuły logicznej (Tabela \ref{Tbl:NDB-logform}).
Z założenia algorytm sprawdzający przynależność do $DB$ sprawdza czy jakakolwiek formuła z~$NDB$ jest spełniana przez dany rekord. 
Dane znajdują się w~$DB$ wtedy i~tylko wtedy gdy żadna formuła nie zostanie spełniona. 

Taki model działania wymusza na danych stałą wielkość,
co jednak nie stanowi problemu w~przypadku przechowywania skrótów haseł które mają stałą, zależną od konkretnego algorytmu długość.
Dla danych o~zmiennych rozmiarach (np. nazwy użytkownika) można zastosować funkcję hashującą lub algorytmy zwiększające długość ciągu bitowego do stałej wartości typu PKCS\#5 lub PKCS\#7.
Należy jednak pamiętać, że zwiększenie długości rekordu znacznie wydłuża czas generacji bazy oraz zajmowaną pamięć dla niektórych algorytmów. 

\begin{table}[h]
    \centering

    \begin{tabular}{|l|l|}
    	\hline
    	rekord NDB & formuła logiczna                      \\ \hline
    	011*       & $ \neg x_1 \land x_2 \land x_3$       \\ \hline
    	001*       & $ \neg x_1 \land \neg x_2 \land x_3 $ \\ \hline
    	1*1*       & $x_1 \land x_3 $                      \\ \hline
    	0*0*       & $\neg x_1 \land \neg x_3 $            \\ \hline
    	1*00       & $x_1 \land \neg x_3 \land  \neg x_4$  \\ \hline
    \end{tabular}
    
    \caption{Reprezentacja formuł logicznych za pomocą NDB kodującej zbiór $DB~=~\{1001,~1101,~1110\}$}
    \label{Tbl:NDB-logform}
\end{table}

\section{Zastosowanie w~systemach uwierzytelniania}
NDB może być wykorzystana w~każdym systemie, gdzie podstawową operacją na danych jest sprawdzenie czy
dany rekord znajduje się w bazie. Jednym z~najpopularniejszych systemów uwierzytelniania jest metoda oparta na loginie i~haśle.
Użytkownik danej aplikacji przy zakładaniu konta podaje hasło, które następnie warstwa serwerowa danej aplikacji przechowuje jako wynik nieodwracalnej funkcji hashującej.

W przypadku nieautoryzowanego dostępu do bazy danych i~używanego algorytmu uzyskiwania skrótu z~hasła, atakujący może uzyskać 
wartość pierwotną mało skomplikowanych haseł za pomocą np. metody słownikowej. Modyfikując powyższy algorytm 
składując skróty jako rekordy w~NDB uniemożliwiamy łatwą iterację wszystkich danych, jednocześnie pozostawiając
łatwy dostęp do informacji czy użytkownik o~podanym loginie i~haśle ma dostęp do aplikacji.

\section{Algorytmy generacji Negatywnych Baz Danych}
\subsection{Algorytm prefiksowy}
\label{sec:prefix-alg}
Najprostszym ze sposobów generowania Negatywnych Baz Danych jest zaproponowany przez Fernando Esponda 
algorytm prefiksowy \cite{NRI-Esponda, Esponda2004EnhancingPT}. Został on opracowany w~celu udowodnienia
że proces generowania NDB z~rekordów $DB$ jest możliwy w~rozsądnej złożoności czasowej i~pamięciowej.
\\\\\\
\begin{algorithm}[H]
    \SetAlgoLined
    \SetAlgorithmName{Algorytm}{}{}
    \KwData{$DB$ - zbiór rekordów do zareprezentowania w~NDB, $l$ - długość rekordu $DB$}
    \KwResult{Zbiór rekordów NDB}
    $Prefix_n(V)$ - Prefiks n-znakowy rekordu $V$\\
    $len(V)$ - Długość rekordu $V$\\
    $W_i$ = \{\};\\
    i = 0;\\
    \While{i < l}{
        $W_{i+1}$ = Zbiór wszystkich $i+1$-znakowych ciągów bitowych $V_p$ nie będących prefiksem żadnego rekordu $DB$ i~dla których $Prefix_i(V_p) \in W_i$\\
        \ForEach{$V_p$ in $W_{i+1}$}{
            Stwórz rekord NDB o~długości $l$ którego $V_p$ jest prefiksem a na pozostałych pozycjach jest symbol $*$ i~dodaj do zbioru wyjściowego NDB
            
        }
        i = i + 1;\\
        $W_i$ = Zbiór wszystkich $i$-znakowych prefiksów rekordów $DB$
    }
    
    \caption{Algorytm prefiksowy}
    \label{alg:prefix}
\end{algorithm}
~\\\\


Powyższa metoda polega na generowaniu coraz dłuższych prefiksów które nie pokrywają się ze zbiorem $DB$.
W ten sposób na początku tworzone są rekordy odpowiadające znacznej części $U~-~DB$. Czasami występuje potrzeba zdefiniowania pewnych rekordów explicite bez wykorzystania symbolu $*$ jeżeli każdy możliwy prefiks jest także prefiksem rekordu z~$DB$. 
Przykładowy wynik działania znajduje się w~tabeli \ref{tbl:prefix_results}.

\begin{table}[h]
    \centering
    \begin{tabular}{|l|l|l|}
        \hline
        DB   & U - DB & NDB  \\ \hline
        0000 & 0001   & 10** \\
        0110 & 0011   & 010* \\
        0010 & 0100   & 111* \\
        1101 & 0101   & 0001 \\
        & 0111   & 0011 \\
        & 1000   & 0111 \\
        & 1001   & 1100 \\
        & 1010   &      \\
        & 1011   &      \\
        & 1100   &      \\
        & 1110   &      \\
        & 1111   &      \\ \hline
    \end{tabular}
    \caption{Rezultat działania algorytmu prefiksowego}
    \label{tbl:prefix_results}
\end{table}



Algorytm prefiksowy jest deterministyczny i~każdy powstały rekord reprezentuje unikalną, nie pokrywającą się część $U~-~DB$ \cite{NRI-Esponda}.
Powoduje to, że algorytm uzyskiwania zbioru $DB$ z~otrzymanej $NDB$ nie wymaga sprowadzenia do problemu SAT. Wystarczy jedynie odpowiednio posortować rekordy i~wyznaczyć przedziały pomiędzy nimi, co pokazuję w algorytmie \ref{alg:prefix-reverse}.

\begin{algorithm}[H]
    \SetAlgoLined
    \SetAlgorithmName{Algorytm}{}{}
    \KwData{$NDB$ - zbiór rekordów do odwrócenia, $l$ - długość rekordu $NDB$}
    \KwResult{Zbiór rekordów DB}
    Posortuj zbiór $NDB$\\
    $V_l$ = rekord $DB$ wypełniony symbolami '$0$' o długości $l$\\
    \ForEach{$V$ \textbf{in} $NDB$}
    {
        $V_h$ = $V$ z zamienionymi symbolami '$*$' na '$0$'\\
        $R$ = $Rekordy\_DB\_W\_Przedziale(V_l, V_h)$\\
        Dodaj $R$ do zbioru wynikowego \\
        $V_l = V$ z zamienionymi symbolami '$*$' na '$1$'\\
    }
    $V_h$ = rekord $DB$ wypełniony symbolami '$1$' o długości $l$\\
    $R$ = $Rekordy\_DB\_W\_Przedziale(V_l, V_h)$\\
    Dodaj $R$ do zbioru wynikowego \\
    \caption{Algorytm odwracający $NDB$ wygenerowaną za pomocą algorytmu prefiksowego}
    \label{alg:prefix-reverse}
\end{algorithm}

\begin{algorithm}[H]
    \SetAlgoLined
    \SetAlgorithmName{Algorytm}{}{}
    \KwData{$V_1, V_2$ - rekordy $DB$}
    \KwResult{Zbiór rekordów DB w przedziale $(V_1, V_2)$}
    Dodaj binarnie $1$ do $V_1$\\
    \While{$V_1 \neq V_2$}
    {
       Dodaj $V_1$ do zbioru wynikowego. \\
       Dodaj binarnie $1$ do $V_1$\\
    }
    \caption{Rekordy\_DB\_W\_Przedziale}
    \label{alg:prefix-reverse-print}
\end{algorithm}


Czas wykonywania procedury generacyjnej wynosi $O(l|DB|)$, jednak w~przypadku zapisywania wyniku do bazy wzrasta do $O(l^2|DB|)$ gdyż konieczna jest serializacja każdego rekordu.
Złożoność pamięciowa dla optymalnej implementacji wynosi $O(l|DB|)$ w~przypadku gdy poprzednio generowane rekordy $NDB$ nie są przetrzymywane w~pamięci. 
Dla danego zbioru $DB$ generowane jest $O(l|DB|)$ rekordów co sprowadza się do wielkości powstałej bazy danych wynoszącej $O(l^2|DB|)$.

Przestawiony powyżej algorytm odwracający (\ref{alg:prefix-reverse}) wykonuje się w~czasie $O(|NDB|)=~O(l|DB|)$ co jest wynikiem lepszym od procedury generacyjnej z zapisem do bazy.

\subsection{Algorytm \textit{Randomize\_NDB}}
Algorytm prefiksowy generuje poprawne rekordy $NDB$, jednak nie jest praktyczny w~żadnych zastosowaniach związanych
z~bezpieczeństwem, ponieważ wynik jego działania jest stosunkowo prosto sprowadzić do postaci pozytywnej. 
Aby temu zaradzić, Fernando Esponda w~swojej pracy\cite{NRI-Esponda} zaproponował 
niedeterministyczny algorytm mający na celu wprowadzić rekordy które nie odpowiadają jedynie prefiksom elementów
z $U - DB$ i~są trudniejsze do odwrócenia.

\begin{algorithm}[H]
    \SetAlgoLined
    \SetAlgorithmName{Algorytm}{}{}
    \KwData{$DB$ - zbiór rekordów do zareprezentowania w~NDB, $l$ - długość rekordu $DB$}
    \KwResult{Zbiór rekordów NDB}
    $Prefix_n(V)$ - Prefiks n-znakowy rekordu $V$\\
    $len(V)$ - Długość rekordu $V$\\
    $\pi$ = losowa permutacja o~długości $|V_{pe}|$\\
    $W_i$ = zbiór wszystkich ciągów $l$-bitowych\\
    $\pi(DB)$ $\equiv$ $\{\pi(V) ~|~ V \in DB\}$\\
    $i$ = $\left \lceil{log_2(l)}\right \rceil$;\\
    \While{$i < l$ \textbf{and} $W_i \neq \emptyset $}{
        $W_{i+1}$ = Zbiór wszystkich $i+1$-znakowych ciągów bitowych $V_p$ nie będących prefiksem żadnego rekordu $\pi(DB)$ i~dla których $Prefix_i(V_p) \in W_i$\\
        \ForEach{$V_p$ \textbf{in} $W_{i+1}$}{
            Dopełnij $V_p$ do długości $l$ wstawiając znaki nieustalone '$*$' na końcu\\ 
            $j$ = losowa liczba z~przedziału $[1, l]$\\
            \For{$k = 1$ \textbf{to} $j$}
            {
                $n$ = losowa liczba z~przedziału $[1, log_2(l)]$\\
                $P$ = $n$ losowych nieustalonych pozycji z~$V_p$\\  
                $X$ = Zbiór rekordów powstałych przez zastąpienie pozycji $\in P$ przez wszystkie możliwe kombinacje bitowe ($2^n$ rekordów)\\
                \ForEach{$V_q$ \textbf{in} $X$}
                {
                    $V_{pg}$ = Pattern\_Generate($\pi(DB)$, $V_q$)\\
                    Dodaj $\pi^{-1}(V_{pg})$ do zbioru rekordów wyjsciowych\\
                }
                
            }
            
        }
        $i = i + 1$\\
        $W_i$ = Zbiór wszystkich $i$-znakowych prefiksów rekordów $DB$
    }    
    \caption{Algorytm \textit{Randomize\_NDB}}
    \label{alg:randomize}
\end{algorithm}
~\\\\
Algorytm ten działa na podobnej zasadzie co algorytm prefiksowy (rozdział \ref{sec:prefix-alg}) z~pewnymi modyfikacjami.
Na początku kolejność bitów w~$DB$ jest mieszana za pomocą losowej permutacji aby pozycje zdefiniowane w~generowanej prefiksowej $NDB$ nie były skumulowane na początku wyrazów. Następnie dla każdego powstałego negatywnego rekordu losuje się $n$ pozycji nieustalonych i~zastępuje się go równoznacznym zbiorem rekordów które mają te pozycje ustalone. 
Powstałe ciągi są dodatkowo zaciemniane przez wstawienie na losowych pozycjach zamiast bitu zdefiniowanego znak $*$ zgodnie z~algorytmem \textit{Pattern\_Generate}.

Wynik algorytmu jest niedeterministyczny co powoduje że dla takich samych zbiorów $DB$ rezultat może się różnić. Generacja wielu redundantnych rekordów zwiększa odporność otrzymanej bazy na próby przywrócenia do postaci pozytywnej, jednak wiąże się to z~ze zwiększeniem objętości NDB średnio $\frac{l^2}{2}$ razy w~stosunku do algorytmu prefiksowego.


\begin{algorithm}[!htb]
    \SetAlgoLined
    \SetAlgorithmName{Algorytm}{}{}
    \KwData{$DB$ - zbiór rekordów do zareprezentowania w~NDB, $V_{q}$ - rekord NDB do zaciemnienia}
    \KwResult{Zaciemniony rekord NDB}
    $\pi$ = losowa permutacja o~długości $|V_{pe}|$\\
    $SIV$ = \{\} // wektor zmienionych bitów\\
    \For{$i = 1$ to $|V_{q}|$}
    {
        $\pi(V_{q})'$ = $\pi(V_{q})$ z~elementem o~indeksie $i$ zamienionym na symbol $*$\\
        \If{istnieje rekord w~$DB$ pokrywający się z~$\pi(V_{q})'$}
        {
            Dodaj $i$ i bit o~indeksie $i$ do $SIV$\\
            $\pi(V_{q}) = \pi(V_{q})'$\\
        }  
    }
    $t =$ losowa liczba z~przedziału $[0, |SIV|]$\\
    \uIf{$t > |SIV|$}
    {
        $R = SIV$
    }
    \Else
    {
        $R = t$ losowych bitów z~$SIV$ 
    }
    $V_k$ = $\pi(V_{q})$\\
    \For{$indeks$, $bit$ in $R$}
    {
        $V_k$[$indeks$] = $bit$
    }
    Zwróć $\pi^{-1}(V_k)$ 
    
    \caption{Algorytm Pattern\_Generate}
    \label{alg:pattern-generate}
\end{algorithm}
~\\\\\\

\newpage
Zmiany względem algorytmu prefiksowego oraz procedura \textit{Pattern\_Generate} gwarantują że żaden ze zmodyfikowanych rekordów nie będzie odpowiadał żadnemu elementowi $DB$ oraz że wynik działania pozostanie kompletnym odzwierciedleniem $U - DB$,
ponieważ każda modyfikacja rekordu rozszerza zakres pokrywanych ciągów wprowadzając redundantne informacje.
W \cite{NRI-Esponda} przedstawiony jest dowód że problem rekonstrukcji $DB$ z~$NDB$ jest NP-trudny (każdą instancję 3-SAT można sprowadzić do $NDB$).

Powyższy algorytm w~teorii jest zdolny do tworzenia trudnych instancji, ale nie posiada żadnych mechanizmów umożliwiających ingerencję w~jego działanie -- wynik jest zawsze losowy.
Prawdopodobieństwo uzyskania trudnej do złamania kombinacji rekordów jest bardzo małe, co powoduje że w znacznej większości przypadków wynikowa $NDB$ może być odwrócona niemal natychmiast przez współczesne solwery SAT, co pokazuję w~dalszych rozdziałach.       


\begin{table}[!htb]
    \centering
    \begin{tabular}{|l|l|l|}
    	\hline
    	DB   & U - DB & NDB  \\ \hline
    	0000 & 0001   & *001 \\
    	0110 & 0011   & 00*1 \\
    	0010 & 0100   & 0011 \\
    	1101 & 0101   & 010* \\
    	     & 0111   & 0*11 \\
    	     & 1000   & 1**0 \\
    	     & 1001   & 10**     \\
    	     & 1010   & 101*    \\
    	     & 1011   & **11     \\
    	     & 1100   & 1*1*     \\
    	     & 1110   & 1100    \\
    	     & 1111   & 111*     \\ \hline
    \end{tabular}
    \caption{Rezultat działania algorytmu \textit{Randomize\_NDB}}
    \label{tbl:randomized_results}
\end{table}

\newpage
\subsection{Algorytm \textit{0-Hidden}}
\label{sec:0hidden}
Następujące algorytmy powstały przez zastosowanie procedur generowania trudnych instancji SAT.
Jedną z~nich jest \textbf{0-Hidden} służąca do generacji formuł 3-SAT nie posiadających rozwiązania, co pozwala na testowanie solwerów 
pod względem wykrywania braku spełnialności \cite{GeneratingHardFormulasByHidingSolutionsDeceptively}.

 ~\\\\
 \begin{algorithm}[H]
     \SetAlgoLined
     \SetAlgorithmName{Algorytm}{}{}
     \KwData{$l$ - liczba zmiennych, $r$ - współczynnik ilości klauzul}
     \KwResult{Zbiór klauzul 3CNF}
     
     $n = l * r$\\
     $W$ = \{\}\\
     \While{$|W| \neq n$}
     {
         Wybierz 3 losowe zmienne\\
         Stwórz klauzulę 3CNF używając wylosowanych zmiennych, z~losowymi znakami\\
         Dodaj klauzulę do zbioru wynikowego $W$
     } 
     
     \caption{Algorytm \textit{0-Hidden}}
     \label{alg:0hidden}
 \end{algorithm}
 ~\\\\
Powyższy algorytm generuje formułę CNF, która z~dużym prawdopodobieństwem nie jest spełnialna i~jest trudna do rozwiązania przez solwery SAT \cite{GeneratingHardFormulasByHidingSolutionsDeceptively, HidingSatisfyingAssignmentsTwoAreBetterThanOne}.
Modyfikując parametr $r$ możemy wpłynąć na rozmiar formuły, wartość $r \approx 4.27$ jest wartością graniczną powyżej której problem prawie na pewno nie ma rozwiązania \cite{GeneratingHardFormulasByHidingSolutionsDeceptively}.

Z perspektywy Negatywnych Baz Danych można przyjąć że powstała formuła odpowiada zbiorowi $U - DB$, jeśli $DB$ jest zbiorem pustym.

\subsection{Algorytmy \textit{1-Hidden} i~\textit{2-Hidden}}
\label{sec:12hidden}
Algorytmy \textit{\textbf{1-Hidden}} i~\textit{\textbf{2-Hidden}} są skonstruowane podobnie jak \textit{0-Hidden}, z~tą różnicą, że formuła wyjściowa ma odpowiednio co najmniej (i w~znacznej większości dokładnie) jedno lub dwa rozwiązania.  
 
W celu testowania możliwości solwerów SAT co do znajdowania przypisania spełniającego daną formułę można generować losowe formuły (Algorytm \ref{alg:0hidden}) i~odrzucać klauzule które
przeczą ukrytemu rozwiązaniu $A$ (\textit{1-Hidden}). Jednak spowoduje to, że rozkład losowy zostanie zaburzony i~solwery mogą to \enquote{poczuć}, co doprowadzi je do ukrytego rozwiązania \cite{HidingSatisfyingAssignmentsTwoAreBetterThanOne}.
    
Aby temu przeciwdziałać można jednocześnie ukryć przypisania $A$ oraz $\neg A$ (\textit{2-Hidden}), co spowoduje że algorytm przeszukujący będzie równoważnie \enquote{przyciągany} przez dwa przeciwne rozwiązania.

Powyższy schemat działania jest wykorzystany w~procedurze \textit{\textbf{Q-Hidden}} będącej rozszerzeniem tego konceptu.


\subsection{Algorytm \textit{Q-Hidden}}
  Metoda \textit{\textbf{Q-Hidden}} do generacji trudnych formuł K-SAT zaproponowana w~\cite{GeneratingHardFormulasByHidingSolutionsDeceptively} rozszerza możliwości algorytmów z~rozdziałów \ref{sec:0hidden} i~\ref{sec:12hidden}
  przez dodanie parametru prawdopodobieństwa $q$ $\in (0, 1)$. Każda wygenerowana klauzula składa się z~$k$ zmiennych których przypisanie pokrywa się z~$t > 0$ literałów z~prawdopodobieństwem $q^t$ (gdzie $t$ to numer kolejnego zgodnego literału).
  Oznacza to że im mniejsza wartość $q$ tym bardziej formuła będzie wskazywać w~przeciwnym kierunku niż ukryte rozwiązanie $A$. 
  
  Na podstawie tego schematu powstał algorytm mający na celu rozwiązanie problemów z~algorytmami prefiksowym i~Randomize\_NDB tj. generowanie zbiorów rekordów które łatwo odwrócić za pomocą solwerów SAT\cite{HARD-NDB}
\\\\\\
 \begin{algorithm}[H]
    \SetAlgoLined
    \SetAlgorithmName{Algorytm}{}{}
    \KwData{$s$ - ciąg bitowy, $l$ - długość rekordu, $r$ - wsp. ilości klauzul\\ $k$ - ilość ustalonych bitów w~klauzuli, $q$ - wsp. prawdopodobieństwa}
    \KwResult{Zbiór rekordów NDB}
    
    $n = l * r$\\
    $NDB = \{\}$\\
    \While{$|NDB| \neq n$}
    {
        $\Upsilon$ = $k$ różnych losowych pozycji w~przedziale $[1, l]$\\
        $V$ = rekord $NDB$ o~długości $l$ wypełniony znakami '$*$'\\
        \ForEach{$i$ \textbf{in} $\Upsilon$}
        {
            $V$[$i$] = $s$[$i$]
        }
        $d$ = 0\\
        \While{$d \neq 1$}
        {
            \ForEach{$i$ \textbf{in} $\Upsilon$}
            {
                $p$ = losowa liczba rzeczywista z~przedziału $[0, 1]$\\ 
                \If{$q > p$}
                {
                    $V$[$i$] = $\neg V$[$i$]\\
                    $d = 1$
                }
            }    
        }
        
    
        Dodaj rekord $V$ do $NDB$
    } 
    
    \caption{Algorytm \textit{Q-Hidden}}
    \label{alg:qhidden}
\end{algorithm}


W~przeciwieństwie do poprzednich metod, za pomocą algorytmu \textit{Q-Hidden} przy zastosowaniu odpowiednich
parametrów $q$, $r$ i~$k$ jest możliwe praktycznie stworzenie instancji $NDB$ rozwiązywalnych przez solwery SAT 
w~czasie potęgowym tj. przez całkowite przeszukanie, co powoduje że dla odpowiednio dużych $l$ odzyskanie 
zabezpieczonego ciągu jest niepraktyczne.

Takie podejście do problemu ma szereg konsekwencji. Pierwszą z~nich jest to że wielkość otrzymanej bazy
nie zależy od przechowywanego rekordu i~można ją kontrolować za pomocą parametru $r$~--~generowane
jest dokładnie $l * r$ rekordów. Nie ma jednak możliwości zawarcia w~jednej instancji $NDB$ 
więcej niż jednego ciągu bitowego - można zakodować jedynie zero lub jeden rekord.

W~\cite{HARD-NDB} razem z~przedstawieniem algorytmu \ref{alg:qhidden} opisany jest przykładowe zastosowanie tego schematu 
w~systemach rzeczywistych wymagających znacznie większych ilości danych. Polega na przechowywaniu zbioru osobnych $NDB$ 
dla każdego pozytywnego rekordu. Metoda zapytania do tak skonstruowanej bazy polega na sprawdzeniu po kolei wszystkich pojedynczych $NDB$.
Jeżeli rekord nie pokrywa się z~żadną formułą w~jakiejkolwiek pod-bazie to przyjmuje się że znajduje się w~bazie głównej.

Negatywne Bazy Danych nie przechowujące żadnych danych okazują się przydatne w~takim rozwiązaniu ponieważ \enquote{puste} bazy stworzone
przez algorytm \textit{0-Hidden} (zmodyfikowany w~celu generowania formuł o~wymaganej liczbie literałów w~klauzuli) są nierozróżnialne
od tych wygenerowanych przez algorytm \textit{Q-Hidden}. Pozwala to na ukrycie liczby rekordów pozytywnych.

\begin{table}[!tb]
    \centering
    \begin{tabular}{|lll|}
        \hline
        \multicolumn{3}{|c|}{NDB} \\ \hline
        *10*1 & 11**0 & **100 \\
        11**0 & **100 & *000* \\
        **100 & *000* & 0*01* \\
        *000* & 0*01* & *0*01 \\
        0*01* & *0*01 & 01*1* \\
        *0*01 & 01*1* & 1*0*1 \\
        01*1* & 1*0*1 & 1**01 \\
        \hline
    \end{tabular}
    \caption{Rezultat działania algorytmu \textit{Q-Hidden} dla ciągu bitowego $10010$ przy parametrach $k=3$, $q=0.5$ i~$r=4.2$}
    \label{tbl:qhidden_results}
\end{table}


Istotnym problemem jest fakt, że nie ma gwarancji, że rezultat działania tego algorytmu będzie pokrywał cały zbiór $U - \{s\}$,
nawet powyżej wartości granicznej $r \approx 4.27$. Jedną z opcji jest dalsze zwiększenie parametru $r$ żeby jeszcze 
bardziej zmniejszyć prawdopodobieństwo zawarcia nadmiarowych ciągów, ale wprowadza to dodatkowe, redundantne informacje co powoduje
że czas potrzebny na odwrócenie $NDB$ maleje. Dlatego wskazane jest  ustawienie niskiej wartości $r$ i~dodanie do każdego rekordu 
dodatkowej informacji wskazującej na jego poprawność np. bitu parzystości, kodu CRC lub wyniku funkcji skrótu.
W takiej sytuacji prawdopodobieństwo wystąpienia niechcianych danych maleje $2^{c}$-krotnie, gdzie $c$ oznacza liczbę dodatkowych bitów.

Rozkład wartości nadmiarowych ciągów bitowych nie jest jednostajny - dodatkowe wystąpienia będą bliskie $s$ w~odległości Hamminga
(tj. poszczególne bity będą się różniły w~niewielkiej ilości), dlatego metoda CRC maksymalizująca odległość Hamminga 
dla podobnych napisów może być szczególnie efektywna. \cite{HARD-NDB}

Porównanie poszczególnych metod przedstawiam w~rozdziale \ref{sec:red-str-test}.

\subsection{Algorytm \textit{K-Hidden}}
\textbf{\textit{K-Hidden}} jest modyfikacją algorytmu \textit{Q-Hidden}. Zamiast pojedynczego parametru prawdopodobieństwa $q$ stosuje się w~nim
wektor parametrów $\{p_1 \dots p_k\}$ w~celu zwiększenia ilości stopni swobody algorytmu \cite{k-hidden}. W~ten sposób możemy dowolnie kontrolować dystrybucję
różnych typów klauzul -~w~zależności od liczby pokrywających się bitów z~ciągiem wejściowym.
\\



 \begin{algorithm}[H]
    \SetAlgoLined
    \SetAlgorithmName{Algorytm}{}{}
    \KwData{$s$ - ciąg bitowy, $l$ - długość rekordu, $r$ - wsp. ilości klauzul\\ $k$ - ilość ustalonych bitów w~klauzuli, $\{p_1 \dots p_k\}$ - wsp. prawdopodobieństwa}
    \KwResult{Zbiór rekordów NDB}
    
    $n = l * r$\\
    $NDB = \{\}$\\
    Q = $\{Q_0, Q_1 \dots Q_k \}$ : $Q_0 = 0, Q_i = p_1 + \dots + p_i $\\
    \While{$|NDB| \neq n$}
    {
        $rnd$ = losowa liczba rzeczywista z~przedziału $(0,1)$\\
        $i = i : Q_{i-1} \le rnd \le Q_i$\\
        $V$ = rekord z $k$ ustalonymi bitami, pokrywający się z $s$ na $k-i$ losowych pozycjach\\
        Dodaj rekord $V$ do $NDB$
    } 
    \caption{Algorytm \textit{K-Hidden}}
    \label{alg:khidden}
\end{algorithm}
~\\
Przed generacją każdego kolejnego rekordu losuje się liczbę z przedziału $(0,1)$ i~na podstawie jej i~wektora parametrów $p_i$ wybiera
się odpowiedni typ rekordu - mający dokładnie $i$~pokrywających się zmiennych gdzie $i$~jest indeksem najmniejszego parametru większego
od wylosowanej liczby.

Rezultat działania ma podobne cechy co wygenerowany przez \textit{Q-Hidden} - możliwość zakodowania tylko
jednego napisu i~brak gwarancji pokrycia całego zbioru $U - \{s\}$.

W~literaturze pojawia się także zaproponowany wcześniej algorytm \textit{p-Hidden} będący szczególnym przypadkiem \textit{K-Hidden} dla $k=3$ \cite{p-hidden}.
Analogicznie dla przypadku gdy wektor parametrów $p$~jest następujący:

\[ p_i = \frac{{{k}\choose{i}}q^i}{(1+ q)^k - 1}, i = 1 \dots k-1   \]

algorytm jest równoważny \textit{Q-Hidden}. \cite{k-hidden}

\subsection{Algorytm hybrydowy}
Algorytm hybrydowy jest połączeniem algorytmu prefiksowego oraz \textit{Q-Hidden} \cite{hybrid}. 
Ma na celu zapewnienie, że wygenerowana baza będzie zawierać całe uniwersum napisów bez jednego, ukrytego rekordu i~jednocześnie
będzie odporna na odwrócenie.  Baza wynikowa jest równoważna formule 3-CNF.

\begin{algorithm}[!htb]
    \SetAlgoLined
    \SetAlgorithmName{Algorytm}{}{}
    \KwData{$s$ - ciąg bitowy, $l$ - długość rekordu, $r$ - wsp. ilości klauzul\\~~~~~~~~~~ $q$ - wsp. prawdopodobieństwa}
    \KwResult{Zbiór rekordów NDB}
   
    $NDB$ = GenComplete($s$, $l$)\\
    $NDB$ = MakeHardReverse($NDB$, $s$, $l$, $r$, $q$)\\
    \caption{Algorytm hybrydowy}
    \label{alg:hybrid}
\end{algorithm}

Proces generacji składa się z~dwóch etapów. Pierwszy, procedura \textit{GenComplete}, tworzy kompletny zbiór $l + 4$ rekordów pokrywający
całe $U - \{s\}$. Na początku tworzy $l-2$ rekordy kolejno zaprzeczając każdemu bitowi oprócz dwóch ostatnich
i~losuje dwie dodatkowe pozycje przed nim ustawiając je zgodnie~z~$s$.
Następnie generuje dwa rekordy dla różniącej się pozycji drugiej i~cztery dla pierwszej.

$NDB$ jest tworzona dla losowej permutacji napisu $s$~a~następnie każdy rekord 
jest przekształcany z~powrotem przez permutację odwrotną w~celu utrudnienia analizy przez pomieszanie kolejności bitów.

W kolejnym etapie tworzone są pozostałe $l*r - (l+4)$ rekordy przez zastosowanie procedury \textit{MakeHardReverse} (Algorytm \ref{alg:hybrid-makehard}) - równoznacznej
\textit{Q-Hidden} dla $k=3$. 

Wynik działania tej metody dla nieznanego napisu $s$ jest nierozróżnialny od baz utworzonych przez algorytmy \textit{0-Hidden}, \textit{1-Hidden},
\textit{2-Hidden}, \textit{Q-Hidden} i~\textit{K-Hidden} przy $k = 3$ i~identycznym $r$. Różnić się będą jedynie rozkładem pozycji
ustalonych zgodnych i~niezgodnych z~ukrytym napisem.

\begin{table}[!tb]
    \centering
    \begin{tabular}{|l|}
    	\hline
    	\textbf{11000} \\ \hline
    	*1*01          \\
    	1*01*          \\
    	111**          \\
    	100**          \\
    	101**          \\
    	0**00          \\
    	0**10          \\
    	0**01          \\
    	0**11          \\ \hline
    \end{tabular}
    \caption{Rezultat działania procedury \textit{GenComplete} dla $\pi(s) = 11000$ przed zastosowaniem permutacji odwrotnej na rekordach}
    \label{tbl:gencomplete_results}
\end{table}


\begin{algorithm}[!htb]
    \SetAlgoLined
    \SetAlgorithmName{Algorytm}{}{}
    \KwData{$s$ - ciąg bitowy, $l$ - długość rekordu}
    \KwResult{Zbiór rekordów NDB}
    $\pi$ = losowa permutacja o~długości $l$\\
    \For{$k=l$ \textbf{to} $3$}
    {
        $V$ = rekord $NDB$ o~długości $l$ zapełniony symbolami '$*$'\\
        $V[k]$ = $\neg \pi(s)[k]$\\
        $i, j$ = dwie różne liczby z przedziału $[1, k]$\\
        $V[i]$ = $\pi(s)[i]$\\
        $V[j]$ = $\pi(s)[j]$\\
        Dodaj rekord $V$ do $NDB$
    }
    $V$ = rekord $NDB$ o~długości $l$ zapełniony symbolami '$*$'\\
    $V[1]$ = $\pi(s)[1]$\\
    $V[2]$ = $\neg \pi(s)[2]$\\
    $i$ = losowa liczba z przedziału $[3, l]$\\
    $V[i]$ = $0$, Dodaj rekord $V$ do $NDB$\\    
    $V[i]$ = $1$, Dodaj rekord $V$ do $NDB$\\

    
    $i$ = losowa liczba z przedziału $[2, l]$\\
    \For{$b = 0$ to $1$}
    {
        $V$ = rekord $NDB$ o~długości $l$ zapełniony symbolami '$*$'\\    
        $V[1]$ = $\neg \pi(s)[1]$\\
        $j$ = losowa liczba z przedziału $[2, l]$, różna od $i$\\
        $V[i]$ = $b$\\
        $V[j]$ = $0$, Dodaj rekord $V$ do $NDB$\\
        $V[j]$ = $1$, Dodaj rekord $V$ do $NDB$\\
    }
    Zwróć $\pi^{-1}(NDB)$
    \caption{GenComplete}
    \label{alg:hybrid-gencomplete}
\end{algorithm}


\begin{algorithm}[H]
    \SetAlgoLined
    \SetAlgorithmName{Algorytm}{}{}
    \KwData{$s$ - ciąg bitowy, $l$ - długość rekordu, $r$ - wsp. ilości klauzul\\~~~~~~~~~~ $q$ - wsp. prawdopodobieństwa}
    \KwResult{Zbiór rekordów $NDB$}
    $n = l * r$\\
    \While{$|NDB| \neq n$}
    {
        $\Upsilon$ = $3$ różne losowe pozycje w~przedziale $[1, l]$\\
        $V$ = rekord $NDB$ o~długości $l$ wypełniony znakami '$*$'\\
        $X$ = losowe przypisanie $\in \{0,1\}^3$\\ 
        
        $u = 3$\\ 
        \ForEach{$i = 1$ \textbf{to} $3$}
        {
            $V[\Upsilon[i]]$ = $X[i]$\\  
            \If{$V[\Upsilon[i]] \neq s[\Upsilon[i]]$}
            {
                $u = u + 1$\\
            }
        }    
        \If{$u \neq 0$}
        {
            $p$ = losowa liczba rzeczywista z przedziału $(0,1)$\\
            \If{$q^u$ > $p$}
            {
                Dodaj rekord $V$ do $NDB$\\
            }    
        }
        
        Dodaj rekord $V$ do $NDB$
    } 
    
    Zwróć powstałą $NDB$\\
    \caption{MakeHardReverse}
    \label{alg:hybrid-makehard}
\end{algorithm}


\section{Wykorzystywanie solwerów SAT w~celu uzyskania przeciwobrazu Negatywnej Bazy Danych}

Problem znalezienia ukrytego ciągu w~Negatywnej Bazie Danych jest równoważny SAT i~zapis napisów nad alfabetem 
$\{0, 1, *\}$ można sprowadzić do CNF w~czasie liniowym.
Dla przykładu, poniższy zbiór rekordów można zinterpretować jako 
\enquote{Znajdź takie przypisanie wartościami \{0, 1\}, żeby dla każdego rekordu nie pokrywało się na co najmniej jednej pozycji}.

\begin{table}[h]
    
    \centering
    \label{Tbl:NDB-sat-example}
    \begin{tabular}{|l|}
        \hline
        11*0 \\ \hline
        001* \\ \hline
        *111 \\ \hline
        *101 \\ \hline
    \end{tabular}
    \caption{Przykładowy zbiór rekordów $NDB$}
\end{table}

Zatem pomijając symbole $*$, zamieniając wartości $\{0,1\}$ na odpowiadające literały, negując je i~łącząc operatorem alternatywy
oraz wstawiając operatory koniunkcji pomiędzy powstałe klauzule otrzymujemy postać CNF:

\[ (\neg x_1 \lor \neg x_2 \lor x_4)~\land \]\[  (x_1 \lor x_2 \lor \neg x_3)~\land \]\[ (\neg x_2 \lor \neg x_3 \lor \neg x_4)~\land \]\[(\neg x_2 \lor x_3 \lor \neg x_4) \]


Metoda działania solwerów SAT nie polega na znalezieniu wszystkich rozwiązań, tylko na sprawdzeniu czy dana formuła jest spełnialna, 
dlatego w~sytuacji próby odwrócenia baz wielorekordowych lub wygenerowanych z~użyciem algorytmów niekompletnych atakujący potrzebuje dla każdego pojedynczego rozwiązania dodać dodatkową formułę przeczącą znalezionemu rekordowi i ponownie uruchomić solwer.   

W związku z~faktem, że uzyskanie dostępu do nawet pojedynczego rekordu w~systemach uwierzytelniania jest niedopuszczalne, testy są przeprowadzone pod kątem znalezienia pierwszego rozwiązania. 